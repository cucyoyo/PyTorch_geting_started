
% Default to the notebook output style

    


% Inherit from the specified cell style.




    
\documentclass[11pt]{article}

    
    
    \usepackage[T1]{fontenc}
    % Nicer default font (+ math font) than Computer Modern for most use cases
    \usepackage{mathpazo}

    % Basic figure setup, for now with no caption control since it's done
    % automatically by Pandoc (which extracts ![](path) syntax from Markdown).
    \usepackage{graphicx}
    % We will generate all images so they have a width \maxwidth. This means
    % that they will get their normal width if they fit onto the page, but
    % are scaled down if they would overflow the margins.
    \makeatletter
    \def\maxwidth{\ifdim\Gin@nat@width>\linewidth\linewidth
    \else\Gin@nat@width\fi}
    \makeatother
    \let\Oldincludegraphics\includegraphics
    % Set max figure width to be 80% of text width, for now hardcoded.
    \renewcommand{\includegraphics}[1]{\Oldincludegraphics[width=.8\maxwidth]{#1}}
    % Ensure that by default, figures have no caption (until we provide a
    % proper Figure object with a Caption API and a way to capture that
    % in the conversion process - todo).
    \usepackage{caption}
    \DeclareCaptionLabelFormat{nolabel}{}
    \captionsetup{labelformat=nolabel}

    \usepackage{adjustbox} % Used to constrain images to a maximum size 
    \usepackage{xcolor} % Allow colors to be defined
    \usepackage{enumerate} % Needed for markdown enumerations to work
    \usepackage{geometry} % Used to adjust the document margins
    \usepackage{amsmath} % Equations
    \usepackage{amssymb} % Equations
    \usepackage{textcomp} % defines textquotesingle
    % Hack from http://tex.stackexchange.com/a/47451/13684:
    \AtBeginDocument{%
        \def\PYZsq{\textquotesingle}% Upright quotes in Pygmentized code
    }
    \usepackage{upquote} % Upright quotes for verbatim code
    \usepackage{eurosym} % defines \euro
    \usepackage[mathletters]{ucs} % Extended unicode (utf-8) support
    \usepackage[utf8x]{inputenc} % Allow utf-8 characters in the tex document
    \usepackage{fancyvrb} % verbatim replacement that allows latex
    \usepackage{grffile} % extends the file name processing of package graphics 
                         % to support a larger range 
    % The hyperref package gives us a pdf with properly built
    % internal navigation ('pdf bookmarks' for the table of contents,
    % internal cross-reference links, web links for URLs, etc.)
    \usepackage{hyperref}
    \usepackage{longtable} % longtable support required by pandoc >1.10
    \usepackage{booktabs}  % table support for pandoc > 1.12.2
    \usepackage[inline]{enumitem} % IRkernel/repr support (it uses the enumerate* environment)
    \usepackage[normalem]{ulem} % ulem is needed to support strikethroughs (\sout)
                                % normalem makes italics be italics, not underlines
    

    
    
    % Colors for the hyperref package
    \definecolor{urlcolor}{rgb}{0,.145,.698}
    \definecolor{linkcolor}{rgb}{.71,0.21,0.01}
    \definecolor{citecolor}{rgb}{.12,.54,.11}

    % ANSI colors
    \definecolor{ansi-black}{HTML}{3E424D}
    \definecolor{ansi-black-intense}{HTML}{282C36}
    \definecolor{ansi-red}{HTML}{E75C58}
    \definecolor{ansi-red-intense}{HTML}{B22B31}
    \definecolor{ansi-green}{HTML}{00A250}
    \definecolor{ansi-green-intense}{HTML}{007427}
    \definecolor{ansi-yellow}{HTML}{DDB62B}
    \definecolor{ansi-yellow-intense}{HTML}{B27D12}
    \definecolor{ansi-blue}{HTML}{208FFB}
    \definecolor{ansi-blue-intense}{HTML}{0065CA}
    \definecolor{ansi-magenta}{HTML}{D160C4}
    \definecolor{ansi-magenta-intense}{HTML}{A03196}
    \definecolor{ansi-cyan}{HTML}{60C6C8}
    \definecolor{ansi-cyan-intense}{HTML}{258F8F}
    \definecolor{ansi-white}{HTML}{C5C1B4}
    \definecolor{ansi-white-intense}{HTML}{A1A6B2}

    % commands and environments needed by pandoc snippets
    % extracted from the output of `pandoc -s`
    \providecommand{\tightlist}{%
      \setlength{\itemsep}{0pt}\setlength{\parskip}{0pt}}
    \DefineVerbatimEnvironment{Highlighting}{Verbatim}{commandchars=\\\{\}}
    % Add ',fontsize=\small' for more characters per line
    \newenvironment{Shaded}{}{}
    \newcommand{\KeywordTok}[1]{\textcolor[rgb]{0.00,0.44,0.13}{\textbf{{#1}}}}
    \newcommand{\DataTypeTok}[1]{\textcolor[rgb]{0.56,0.13,0.00}{{#1}}}
    \newcommand{\DecValTok}[1]{\textcolor[rgb]{0.25,0.63,0.44}{{#1}}}
    \newcommand{\BaseNTok}[1]{\textcolor[rgb]{0.25,0.63,0.44}{{#1}}}
    \newcommand{\FloatTok}[1]{\textcolor[rgb]{0.25,0.63,0.44}{{#1}}}
    \newcommand{\CharTok}[1]{\textcolor[rgb]{0.25,0.44,0.63}{{#1}}}
    \newcommand{\StringTok}[1]{\textcolor[rgb]{0.25,0.44,0.63}{{#1}}}
    \newcommand{\CommentTok}[1]{\textcolor[rgb]{0.38,0.63,0.69}{\textit{{#1}}}}
    \newcommand{\OtherTok}[1]{\textcolor[rgb]{0.00,0.44,0.13}{{#1}}}
    \newcommand{\AlertTok}[1]{\textcolor[rgb]{1.00,0.00,0.00}{\textbf{{#1}}}}
    \newcommand{\FunctionTok}[1]{\textcolor[rgb]{0.02,0.16,0.49}{{#1}}}
    \newcommand{\RegionMarkerTok}[1]{{#1}}
    \newcommand{\ErrorTok}[1]{\textcolor[rgb]{1.00,0.00,0.00}{\textbf{{#1}}}}
    \newcommand{\NormalTok}[1]{{#1}}
    
    % Additional commands for more recent versions of Pandoc
    \newcommand{\ConstantTok}[1]{\textcolor[rgb]{0.53,0.00,0.00}{{#1}}}
    \newcommand{\SpecialCharTok}[1]{\textcolor[rgb]{0.25,0.44,0.63}{{#1}}}
    \newcommand{\VerbatimStringTok}[1]{\textcolor[rgb]{0.25,0.44,0.63}{{#1}}}
    \newcommand{\SpecialStringTok}[1]{\textcolor[rgb]{0.73,0.40,0.53}{{#1}}}
    \newcommand{\ImportTok}[1]{{#1}}
    \newcommand{\DocumentationTok}[1]{\textcolor[rgb]{0.73,0.13,0.13}{\textit{{#1}}}}
    \newcommand{\AnnotationTok}[1]{\textcolor[rgb]{0.38,0.63,0.69}{\textbf{\textit{{#1}}}}}
    \newcommand{\CommentVarTok}[1]{\textcolor[rgb]{0.38,0.63,0.69}{\textbf{\textit{{#1}}}}}
    \newcommand{\VariableTok}[1]{\textcolor[rgb]{0.10,0.09,0.49}{{#1}}}
    \newcommand{\ControlFlowTok}[1]{\textcolor[rgb]{0.00,0.44,0.13}{\textbf{{#1}}}}
    \newcommand{\OperatorTok}[1]{\textcolor[rgb]{0.40,0.40,0.40}{{#1}}}
    \newcommand{\BuiltInTok}[1]{{#1}}
    \newcommand{\ExtensionTok}[1]{{#1}}
    \newcommand{\PreprocessorTok}[1]{\textcolor[rgb]{0.74,0.48,0.00}{{#1}}}
    \newcommand{\AttributeTok}[1]{\textcolor[rgb]{0.49,0.56,0.16}{{#1}}}
    \newcommand{\InformationTok}[1]{\textcolor[rgb]{0.38,0.63,0.69}{\textbf{\textit{{#1}}}}}
    \newcommand{\WarningTok}[1]{\textcolor[rgb]{0.38,0.63,0.69}{\textbf{\textit{{#1}}}}}
    
    
    % Define a nice break command that doesn't care if a line doesn't already
    % exist.
    \def\br{\hspace*{\fill} \\* }
    % Math Jax compatability definitions
    \def\gt{>}
    \def\lt{<}
    % Document parameters
    \title{PyTorch}
    
    
    

    % Pygments definitions
    
\makeatletter
\def\PY@reset{\let\PY@it=\relax \let\PY@bf=\relax%
    \let\PY@ul=\relax \let\PY@tc=\relax%
    \let\PY@bc=\relax \let\PY@ff=\relax}
\def\PY@tok#1{\csname PY@tok@#1\endcsname}
\def\PY@toks#1+{\ifx\relax#1\empty\else%
    \PY@tok{#1}\expandafter\PY@toks\fi}
\def\PY@do#1{\PY@bc{\PY@tc{\PY@ul{%
    \PY@it{\PY@bf{\PY@ff{#1}}}}}}}
\def\PY#1#2{\PY@reset\PY@toks#1+\relax+\PY@do{#2}}

\expandafter\def\csname PY@tok@c\endcsname{\let\PY@it=\textit\def\PY@tc##1{\textcolor[rgb]{0.25,0.50,0.50}{##1}}}
\expandafter\def\csname PY@tok@vm\endcsname{\def\PY@tc##1{\textcolor[rgb]{0.10,0.09,0.49}{##1}}}
\expandafter\def\csname PY@tok@sb\endcsname{\def\PY@tc##1{\textcolor[rgb]{0.73,0.13,0.13}{##1}}}
\expandafter\def\csname PY@tok@fm\endcsname{\def\PY@tc##1{\textcolor[rgb]{0.00,0.00,1.00}{##1}}}
\expandafter\def\csname PY@tok@k\endcsname{\let\PY@bf=\textbf\def\PY@tc##1{\textcolor[rgb]{0.00,0.50,0.00}{##1}}}
\expandafter\def\csname PY@tok@sd\endcsname{\let\PY@it=\textit\def\PY@tc##1{\textcolor[rgb]{0.73,0.13,0.13}{##1}}}
\expandafter\def\csname PY@tok@ch\endcsname{\let\PY@it=\textit\def\PY@tc##1{\textcolor[rgb]{0.25,0.50,0.50}{##1}}}
\expandafter\def\csname PY@tok@ni\endcsname{\let\PY@bf=\textbf\def\PY@tc##1{\textcolor[rgb]{0.60,0.60,0.60}{##1}}}
\expandafter\def\csname PY@tok@nb\endcsname{\def\PY@tc##1{\textcolor[rgb]{0.00,0.50,0.00}{##1}}}
\expandafter\def\csname PY@tok@sx\endcsname{\def\PY@tc##1{\textcolor[rgb]{0.00,0.50,0.00}{##1}}}
\expandafter\def\csname PY@tok@s\endcsname{\def\PY@tc##1{\textcolor[rgb]{0.73,0.13,0.13}{##1}}}
\expandafter\def\csname PY@tok@se\endcsname{\let\PY@bf=\textbf\def\PY@tc##1{\textcolor[rgb]{0.73,0.40,0.13}{##1}}}
\expandafter\def\csname PY@tok@nd\endcsname{\def\PY@tc##1{\textcolor[rgb]{0.67,0.13,1.00}{##1}}}
\expandafter\def\csname PY@tok@cpf\endcsname{\let\PY@it=\textit\def\PY@tc##1{\textcolor[rgb]{0.25,0.50,0.50}{##1}}}
\expandafter\def\csname PY@tok@vi\endcsname{\def\PY@tc##1{\textcolor[rgb]{0.10,0.09,0.49}{##1}}}
\expandafter\def\csname PY@tok@kp\endcsname{\def\PY@tc##1{\textcolor[rgb]{0.00,0.50,0.00}{##1}}}
\expandafter\def\csname PY@tok@ne\endcsname{\let\PY@bf=\textbf\def\PY@tc##1{\textcolor[rgb]{0.82,0.25,0.23}{##1}}}
\expandafter\def\csname PY@tok@dl\endcsname{\def\PY@tc##1{\textcolor[rgb]{0.73,0.13,0.13}{##1}}}
\expandafter\def\csname PY@tok@na\endcsname{\def\PY@tc##1{\textcolor[rgb]{0.49,0.56,0.16}{##1}}}
\expandafter\def\csname PY@tok@gr\endcsname{\def\PY@tc##1{\textcolor[rgb]{1.00,0.00,0.00}{##1}}}
\expandafter\def\csname PY@tok@nl\endcsname{\def\PY@tc##1{\textcolor[rgb]{0.63,0.63,0.00}{##1}}}
\expandafter\def\csname PY@tok@s1\endcsname{\def\PY@tc##1{\textcolor[rgb]{0.73,0.13,0.13}{##1}}}
\expandafter\def\csname PY@tok@cs\endcsname{\let\PY@it=\textit\def\PY@tc##1{\textcolor[rgb]{0.25,0.50,0.50}{##1}}}
\expandafter\def\csname PY@tok@il\endcsname{\def\PY@tc##1{\textcolor[rgb]{0.40,0.40,0.40}{##1}}}
\expandafter\def\csname PY@tok@gd\endcsname{\def\PY@tc##1{\textcolor[rgb]{0.63,0.00,0.00}{##1}}}
\expandafter\def\csname PY@tok@vc\endcsname{\def\PY@tc##1{\textcolor[rgb]{0.10,0.09,0.49}{##1}}}
\expandafter\def\csname PY@tok@kr\endcsname{\let\PY@bf=\textbf\def\PY@tc##1{\textcolor[rgb]{0.00,0.50,0.00}{##1}}}
\expandafter\def\csname PY@tok@cp\endcsname{\def\PY@tc##1{\textcolor[rgb]{0.74,0.48,0.00}{##1}}}
\expandafter\def\csname PY@tok@bp\endcsname{\def\PY@tc##1{\textcolor[rgb]{0.00,0.50,0.00}{##1}}}
\expandafter\def\csname PY@tok@c1\endcsname{\let\PY@it=\textit\def\PY@tc##1{\textcolor[rgb]{0.25,0.50,0.50}{##1}}}
\expandafter\def\csname PY@tok@sh\endcsname{\def\PY@tc##1{\textcolor[rgb]{0.73,0.13,0.13}{##1}}}
\expandafter\def\csname PY@tok@mh\endcsname{\def\PY@tc##1{\textcolor[rgb]{0.40,0.40,0.40}{##1}}}
\expandafter\def\csname PY@tok@gu\endcsname{\let\PY@bf=\textbf\def\PY@tc##1{\textcolor[rgb]{0.50,0.00,0.50}{##1}}}
\expandafter\def\csname PY@tok@gp\endcsname{\let\PY@bf=\textbf\def\PY@tc##1{\textcolor[rgb]{0.00,0.00,0.50}{##1}}}
\expandafter\def\csname PY@tok@nt\endcsname{\let\PY@bf=\textbf\def\PY@tc##1{\textcolor[rgb]{0.00,0.50,0.00}{##1}}}
\expandafter\def\csname PY@tok@mf\endcsname{\def\PY@tc##1{\textcolor[rgb]{0.40,0.40,0.40}{##1}}}
\expandafter\def\csname PY@tok@gt\endcsname{\def\PY@tc##1{\textcolor[rgb]{0.00,0.27,0.87}{##1}}}
\expandafter\def\csname PY@tok@nf\endcsname{\def\PY@tc##1{\textcolor[rgb]{0.00,0.00,1.00}{##1}}}
\expandafter\def\csname PY@tok@kd\endcsname{\let\PY@bf=\textbf\def\PY@tc##1{\textcolor[rgb]{0.00,0.50,0.00}{##1}}}
\expandafter\def\csname PY@tok@sr\endcsname{\def\PY@tc##1{\textcolor[rgb]{0.73,0.40,0.53}{##1}}}
\expandafter\def\csname PY@tok@ow\endcsname{\let\PY@bf=\textbf\def\PY@tc##1{\textcolor[rgb]{0.67,0.13,1.00}{##1}}}
\expandafter\def\csname PY@tok@vg\endcsname{\def\PY@tc##1{\textcolor[rgb]{0.10,0.09,0.49}{##1}}}
\expandafter\def\csname PY@tok@err\endcsname{\def\PY@bc##1{\setlength{\fboxsep}{0pt}\fcolorbox[rgb]{1.00,0.00,0.00}{1,1,1}{\strut ##1}}}
\expandafter\def\csname PY@tok@kc\endcsname{\let\PY@bf=\textbf\def\PY@tc##1{\textcolor[rgb]{0.00,0.50,0.00}{##1}}}
\expandafter\def\csname PY@tok@nc\endcsname{\let\PY@bf=\textbf\def\PY@tc##1{\textcolor[rgb]{0.00,0.00,1.00}{##1}}}
\expandafter\def\csname PY@tok@si\endcsname{\let\PY@bf=\textbf\def\PY@tc##1{\textcolor[rgb]{0.73,0.40,0.53}{##1}}}
\expandafter\def\csname PY@tok@mo\endcsname{\def\PY@tc##1{\textcolor[rgb]{0.40,0.40,0.40}{##1}}}
\expandafter\def\csname PY@tok@ge\endcsname{\let\PY@it=\textit}
\expandafter\def\csname PY@tok@cm\endcsname{\let\PY@it=\textit\def\PY@tc##1{\textcolor[rgb]{0.25,0.50,0.50}{##1}}}
\expandafter\def\csname PY@tok@nn\endcsname{\let\PY@bf=\textbf\def\PY@tc##1{\textcolor[rgb]{0.00,0.00,1.00}{##1}}}
\expandafter\def\csname PY@tok@gi\endcsname{\def\PY@tc##1{\textcolor[rgb]{0.00,0.63,0.00}{##1}}}
\expandafter\def\csname PY@tok@nv\endcsname{\def\PY@tc##1{\textcolor[rgb]{0.10,0.09,0.49}{##1}}}
\expandafter\def\csname PY@tok@mi\endcsname{\def\PY@tc##1{\textcolor[rgb]{0.40,0.40,0.40}{##1}}}
\expandafter\def\csname PY@tok@go\endcsname{\def\PY@tc##1{\textcolor[rgb]{0.53,0.53,0.53}{##1}}}
\expandafter\def\csname PY@tok@sa\endcsname{\def\PY@tc##1{\textcolor[rgb]{0.73,0.13,0.13}{##1}}}
\expandafter\def\csname PY@tok@gh\endcsname{\let\PY@bf=\textbf\def\PY@tc##1{\textcolor[rgb]{0.00,0.00,0.50}{##1}}}
\expandafter\def\csname PY@tok@no\endcsname{\def\PY@tc##1{\textcolor[rgb]{0.53,0.00,0.00}{##1}}}
\expandafter\def\csname PY@tok@mb\endcsname{\def\PY@tc##1{\textcolor[rgb]{0.40,0.40,0.40}{##1}}}
\expandafter\def\csname PY@tok@o\endcsname{\def\PY@tc##1{\textcolor[rgb]{0.40,0.40,0.40}{##1}}}
\expandafter\def\csname PY@tok@kt\endcsname{\def\PY@tc##1{\textcolor[rgb]{0.69,0.00,0.25}{##1}}}
\expandafter\def\csname PY@tok@m\endcsname{\def\PY@tc##1{\textcolor[rgb]{0.40,0.40,0.40}{##1}}}
\expandafter\def\csname PY@tok@sc\endcsname{\def\PY@tc##1{\textcolor[rgb]{0.73,0.13,0.13}{##1}}}
\expandafter\def\csname PY@tok@gs\endcsname{\let\PY@bf=\textbf}
\expandafter\def\csname PY@tok@w\endcsname{\def\PY@tc##1{\textcolor[rgb]{0.73,0.73,0.73}{##1}}}
\expandafter\def\csname PY@tok@s2\endcsname{\def\PY@tc##1{\textcolor[rgb]{0.73,0.13,0.13}{##1}}}
\expandafter\def\csname PY@tok@ss\endcsname{\def\PY@tc##1{\textcolor[rgb]{0.10,0.09,0.49}{##1}}}
\expandafter\def\csname PY@tok@kn\endcsname{\let\PY@bf=\textbf\def\PY@tc##1{\textcolor[rgb]{0.00,0.50,0.00}{##1}}}

\def\PYZbs{\char`\\}
\def\PYZus{\char`\_}
\def\PYZob{\char`\{}
\def\PYZcb{\char`\}}
\def\PYZca{\char`\^}
\def\PYZam{\char`\&}
\def\PYZlt{\char`\<}
\def\PYZgt{\char`\>}
\def\PYZsh{\char`\#}
\def\PYZpc{\char`\%}
\def\PYZdl{\char`\$}
\def\PYZhy{\char`\-}
\def\PYZsq{\char`\'}
\def\PYZdq{\char`\"}
\def\PYZti{\char`\~}
% for compatibility with earlier versions
\def\PYZat{@}
\def\PYZlb{[}
\def\PYZrb{]}
\makeatother


    % Exact colors from NB
    \definecolor{incolor}{rgb}{0.0, 0.0, 0.5}
    \definecolor{outcolor}{rgb}{0.545, 0.0, 0.0}



    
    % Prevent overflowing lines due to hard-to-break entities
    \sloppy 
    % Setup hyperref package
    \hypersetup{
      breaklinks=true,  % so long urls are correctly broken across lines
      colorlinks=true,
      urlcolor=urlcolor,
      linkcolor=linkcolor,
      citecolor=citecolor,
      }
    % Slightly bigger margins than the latex defaults
    
    \geometry{verbose,tmargin=1in,bmargin=1in,lmargin=1in,rmargin=1in}
    
    

    \begin{document}
    
    
    \maketitle
    
    

    
    \subsection{关系拟合(回归)}\label{ux5173ux7cfbux62dfux5408ux56deux5f52}

    \paragraph{建立数据集}\label{ux5efaux7acbux6570ux636eux96c6}

    \begin{Verbatim}[commandchars=\\\{\}]
{\color{incolor}In [{\color{incolor}5}]:} \PY{k+kn}{import} \PY{n+nn}{torch}
        \PY{k+kn}{from} \PY{n+nn}{torch}\PY{n+nn}{.}\PY{n+nn}{autograd} \PY{k}{import} \PY{n}{Variable}
        \PY{k+kn}{import} \PY{n+nn}{matplotlib}\PY{n+nn}{.}\PY{n+nn}{pyplot} \PY{k}{as} \PY{n+nn}{plt}
        
        \PY{c+c1}{\PYZsh{} unsqueeze把以为数据变成二维,torch只能处理二维数据}
        \PY{n}{x} \PY{o}{=} \PY{n}{torch}\PY{o}{.}\PY{n}{unsqueeze}\PY{p}{(}\PY{n}{torch}\PY{o}{.}\PY{n}{linspace}\PY{p}{(}\PY{o}{\PYZhy{}}\PY{l+m+mi}{1}\PY{p}{,} \PY{l+m+mi}{1}\PY{p}{,} \PY{l+m+mi}{100}\PY{p}{)}\PY{p}{,} \PY{n}{dim}\PY{o}{=}\PY{l+m+mi}{1}\PY{p}{)}  \PY{c+c1}{\PYZsh{} x data (tensor), shape=(100, 1)}
        \PY{c+c1}{\PYZsh{} y是x二次方+噪点}
        \PY{n}{y} \PY{o}{=} \PY{n}{x}\PY{o}{.}\PY{n}{pow}\PY{p}{(}\PY{l+m+mi}{2}\PY{p}{)} \PY{o}{+} \PY{l+m+mf}{0.2}\PY{o}{*}\PY{n}{torch}\PY{o}{.}\PY{n}{rand}\PY{p}{(}\PY{n}{x}\PY{o}{.}\PY{n}{size}\PY{p}{(}\PY{p}{)}\PY{p}{)}                 \PY{c+c1}{\PYZsh{} noisy y data (tensor), shape=(100, 1)}
        
        \PY{c+c1}{\PYZsh{} 用 Variable 来修饰这些数据 tensor}
        \PY{n}{x}\PY{p}{,} \PY{n}{y} \PY{o}{=} \PY{n}{torch}\PY{o}{.}\PY{n}{autograd}\PY{o}{.}\PY{n}{Variable}\PY{p}{(}\PY{n}{x}\PY{p}{)}\PY{p}{,} \PY{n}{Variable}\PY{p}{(}\PY{n}{y}\PY{p}{)}
        
        \PY{c+c1}{\PYZsh{} 画图}
        \PY{n}{plt}\PY{o}{.}\PY{n}{scatter}\PY{p}{(}\PY{n}{x}\PY{o}{.}\PY{n}{data}\PY{o}{.}\PY{n}{numpy}\PY{p}{(}\PY{p}{)}\PY{p}{,} \PY{n}{y}\PY{o}{.}\PY{n}{data}\PY{o}{.}\PY{n}{numpy}\PY{p}{(}\PY{p}{)}\PY{p}{)}
        \PY{n}{plt}\PY{o}{.}\PY{n}{show}\PY{p}{(}\PY{p}{)}
\end{Verbatim}


    \begin{center}
    \adjustimage{max size={0.9\linewidth}{0.9\paperheight}}{output_2_0.png}
    \end{center}
    { \hspace*{\fill} \\}
    
    \paragraph{创建神经网络的通用模式}\label{ux521bux5efaux795eux7ecfux7f51ux7edcux7684ux901aux7528ux6a21ux5f0f}

    \begin{Verbatim}[commandchars=\\\{\}]
{\color{incolor}In [{\color{incolor} }]:} \PY{k}{class} \PY{n+nc}{Net}\PY{p}{(}\PY{n}{torch}\PY{o}{.}\PY{n}{nn}\PY{o}{.}\PY{n}{Module}\PY{p}{)}\PY{p}{:}  \PY{c+c1}{\PYZsh{} 继承 torch 的 Module}
            \PY{k}{def} \PY{n+nf}{\PYZus{}\PYZus{}init\PYZus{}\PYZus{}}\PY{p}{(}\PY{n+nb+bp}{self}\PY{p}{)}\PY{p}{:}
                \PY{n+nb}{super}\PY{p}{(}\PY{n}{Net}\PY{p}{,} \PY{n+nb+bp}{self}\PY{p}{)}\PY{o}{.}\PY{n+nf+fm}{\PYZus{}\PYZus{}init\PYZus{}\PYZus{}}\PY{p}{(}\PY{p}{)}     \PY{c+c1}{\PYZsh{} 继承 \PYZus{}\PYZus{}init\PYZus{}\PYZus{} 功能}
                
            \PY{k}{def} \PY{n+nf}{forward}\PY{p}{(}\PY{n+nb+bp}{self}\PY{p}{,} \PY{n}{x}\PY{p}{)}\PY{p}{:}   \PY{c+c1}{\PYZsh{} 这同时也是 Module 中的 forward 功能}
                \PY{k}{pass}
\end{Verbatim}


    \paragraph{建立神经网路 +
训练神经网络}\label{ux5efaux7acbux795eux7ecfux7f51ux8def-ux8badux7ec3ux795eux7ecfux7f51ux7edc}

    \begin{Verbatim}[commandchars=\\\{\}]
{\color{incolor}In [{\color{incolor} }]:} \PY{k+kn}{import} \PY{n+nn}{torch}
        \PY{k+kn}{from} \PY{n+nn}{torch}\PY{n+nn}{.}\PY{n+nn}{autograd} \PY{k}{import} \PY{n}{Variable}
        \PY{k+kn}{import} \PY{n+nn}{torch}\PY{n+nn}{.}\PY{n+nn}{nn}\PY{n+nn}{.}\PY{n+nn}{functional} \PY{k}{as} \PY{n+nn}{F}     \PY{c+c1}{\PYZsh{} 激励函数都在这}
        \PY{k+kn}{import} \PY{n+nn}{matplotlib}\PY{n+nn}{.}\PY{n+nn}{pyplot} \PY{k}{as} \PY{n+nn}{plt}
        
        
        \PY{c+c1}{\PYZsh{}\PYZsh{}\PYZsh{}\PYZsh{} 建立数据集}
        \PY{c+c1}{\PYZsh{} unsqueeze把以为数据变成二维,torch只能处理二维数据}
        \PY{n}{x} \PY{o}{=} \PY{n}{torch}\PY{o}{.}\PY{n}{unsqueeze}\PY{p}{(}\PY{n}{torch}\PY{o}{.}\PY{n}{linspace}\PY{p}{(}\PY{o}{\PYZhy{}}\PY{l+m+mi}{1}\PY{p}{,} \PY{l+m+mi}{1}\PY{p}{,} \PY{l+m+mi}{100}\PY{p}{)}\PY{p}{,} \PY{n}{dim}\PY{o}{=}\PY{l+m+mi}{1}\PY{p}{)}  \PY{c+c1}{\PYZsh{} x data (tensor), shape=(100, 1)}
        \PY{c+c1}{\PYZsh{} y是x二次方+噪点}
        \PY{n}{y} \PY{o}{=} \PY{n}{x}\PY{o}{.}\PY{n}{pow}\PY{p}{(}\PY{l+m+mi}{2}\PY{p}{)} \PY{o}{+} \PY{l+m+mf}{0.2}\PY{o}{*}\PY{n}{torch}\PY{o}{.}\PY{n}{rand}\PY{p}{(}\PY{n}{x}\PY{o}{.}\PY{n}{size}\PY{p}{(}\PY{p}{)}\PY{p}{)}                 \PY{c+c1}{\PYZsh{} noisy y data (tensor), shape=(100, 1)}
        
        \PY{c+c1}{\PYZsh{} 用 Variable 来修饰这些数据 tensor}
        \PY{n}{x}\PY{p}{,} \PY{n}{y} \PY{o}{=} \PY{n}{torch}\PY{o}{.}\PY{n}{autograd}\PY{o}{.}\PY{n}{Variable}\PY{p}{(}\PY{n}{x}\PY{p}{)}\PY{p}{,} \PY{n}{Variable}\PY{p}{(}\PY{n}{y}\PY{p}{)}
        
        
        \PY{c+c1}{\PYZsh{}\PYZsh{}\PYZsh{}\PYZsh{} 建立神经网络}
        \PY{k}{class} \PY{n+nc}{Net}\PY{p}{(}\PY{n}{torch}\PY{o}{.}\PY{n}{nn}\PY{o}{.}\PY{n}{Module}\PY{p}{)}\PY{p}{:}  \PY{c+c1}{\PYZsh{} 继承 torch 的 Module}
            \PY{c+c1}{\PYZsh{}\PYZsh{}\PYZsh{}\PYZsh{}\PYZsh{} 定义}
            \PY{k}{def} \PY{n+nf}{\PYZus{}\PYZus{}init\PYZus{}\PYZus{}}\PY{p}{(}\PY{n+nb+bp}{self}\PY{p}{,} \PY{n}{n\PYZus{}feature}\PY{p}{,} \PY{n}{n\PYZus{}hidden}\PY{p}{,} \PY{n}{n\PYZus{}output}\PY{p}{)}\PY{p}{:}\PY{c+c1}{\PYZsh{} n\PYZus{}feature, n\PYZus{}hidden, n\PYZus{}output分别是输入层、隐藏层、输出层}
                \PY{n+nb}{super}\PY{p}{(}\PY{n}{Net}\PY{p}{,} \PY{n+nb+bp}{self}\PY{p}{)}\PY{o}{.}\PY{n+nf+fm}{\PYZus{}\PYZus{}init\PYZus{}\PYZus{}}\PY{p}{(}\PY{p}{)}     \PY{c+c1}{\PYZsh{} 继承 \PYZus{}\PYZus{}init\PYZus{}\PYZus{} 功能}
                \PY{c+c1}{\PYZsh{} 定义每层用什么样的形式}
                \PY{n+nb+bp}{self}\PY{o}{.}\PY{n}{hidden} \PY{o}{=} \PY{n}{torch}\PY{o}{.}\PY{n}{nn}\PY{o}{.}\PY{n}{Linear}\PY{p}{(}\PY{n}{n\PYZus{}feature}\PY{p}{,} \PY{n}{n\PYZus{}hidden}\PY{p}{)}   \PY{c+c1}{\PYZsh{} 隐藏层线性输出}
                \PY{n+nb+bp}{self}\PY{o}{.}\PY{n}{predict} \PY{o}{=} \PY{n}{torch}\PY{o}{.}\PY{n}{nn}\PY{o}{.}\PY{n}{Linear}\PY{p}{(}\PY{n}{n\PYZus{}hidden}\PY{p}{,} \PY{n}{n\PYZus{}output}\PY{p}{)}   \PY{c+c1}{\PYZsh{} 输出层线性输出}
            \PY{c+c1}{\PYZsh{}\PYZsh{}\PYZsh{}\PYZsh{} 搭建}
            \PY{k}{def} \PY{n+nf}{forward}\PY{p}{(}\PY{n+nb+bp}{self}\PY{p}{,} \PY{n}{x}\PY{p}{)}\PY{p}{:}   \PY{c+c1}{\PYZsh{} 这同时也是 Module 中的 forward 功能}
                \PY{c+c1}{\PYZsh{} 正向传播输入值, 神经网络分析出输出值}
                \PY{n}{x} \PY{o}{=} \PY{n}{F}\PY{o}{.}\PY{n}{relu}\PY{p}{(}\PY{n+nb+bp}{self}\PY{o}{.}\PY{n}{hidden}\PY{p}{(}\PY{n}{x}\PY{p}{)}\PY{p}{)}      \PY{c+c1}{\PYZsh{} 激励函数(隐藏层的线性值)}
                \PY{c+c1}{\PYZsh{} 预测时这里不用激励函数}
                \PY{n}{x} \PY{o}{=} \PY{n+nb+bp}{self}\PY{o}{.}\PY{n}{predict}\PY{p}{(}\PY{n}{x}\PY{p}{)}             \PY{c+c1}{\PYZsh{} 输出值}
                \PY{k}{return} \PY{n}{x}
        
        \PY{n}{net} \PY{o}{=} \PY{n}{Net}\PY{p}{(}\PY{n}{n\PYZus{}feature}\PY{o}{=}\PY{l+m+mi}{1}\PY{p}{,} \PY{n}{n\PYZus{}hidden}\PY{o}{=}\PY{l+m+mi}{10}\PY{p}{,} \PY{n}{n\PYZus{}output}\PY{o}{=}\PY{l+m+mi}{1}\PY{p}{)}\PY{c+c1}{\PYZsh{} 输入值:1;隐藏层:10;输出层:1}
        
        \PY{n+nb}{print}\PY{p}{(}\PY{n}{net}\PY{p}{)}  \PY{c+c1}{\PYZsh{} net 的结构}
        
        
        \PY{n}{plt}\PY{o}{.}\PY{n}{ion}\PY{p}{(}\PY{p}{)}   \PY{c+c1}{\PYZsh{} 设置plt为实时打印的方式}
        \PY{n}{plt}\PY{o}{.}\PY{n}{show}\PY{p}{(}\PY{p}{)}
        
        
        \PY{c+c1}{\PYZsh{}\PYZsh{}\PYZsh{}\PYZsh{} 训练网络}
        \PY{c+c1}{\PYZsh{} optimizer 是训练的工具}
        \PY{n}{optimizer} \PY{o}{=} \PY{n}{torch}\PY{o}{.}\PY{n}{optim}\PY{o}{.}\PY{n}{SGD}\PY{p}{(}\PY{n}{net}\PY{o}{.}\PY{n}{parameters}\PY{p}{(}\PY{p}{)}\PY{p}{,} \PY{n}{lr}\PY{o}{=}\PY{l+m+mf}{0.5}\PY{p}{)}  \PY{c+c1}{\PYZsh{} 优化器: 传入 net 的所有参数, lr是学习率}
        \PY{n}{loss\PYZus{}func} \PY{o}{=} \PY{n}{torch}\PY{o}{.}\PY{n}{nn}\PY{o}{.}\PY{n}{MSELoss}\PY{p}{(}\PY{p}{)}      \PY{c+c1}{\PYZsh{} 预测值和真实值的误差计算公式 (均方差)}
        
        \PY{k}{for} \PY{n}{t} \PY{o+ow}{in} \PY{n+nb}{range}\PY{p}{(}\PY{l+m+mi}{100}\PY{p}{)}\PY{p}{:}
            \PY{n}{prediction} \PY{o}{=} \PY{n}{net}\PY{p}{(}\PY{n}{x}\PY{p}{)}     \PY{c+c1}{\PYZsh{} 喂给 net 训练数据 x, 输出预测值}
        
            \PY{n}{loss} \PY{o}{=} \PY{n}{loss\PYZus{}func}\PY{p}{(}\PY{n}{prediction}\PY{p}{,} \PY{n}{y}\PY{p}{)}     \PY{c+c1}{\PYZsh{} 计算两者的误差;prediction在前,真实值y在后}
        
            \PY{n}{optimizer}\PY{o}{.}\PY{n}{zero\PYZus{}grad}\PY{p}{(}\PY{p}{)}   \PY{c+c1}{\PYZsh{} 清空上一步的残余更新参数值;(将optimizer的梯度设为0)}
            \PY{n}{loss}\PY{o}{.}\PY{n}{backward}\PY{p}{(}\PY{p}{)}         \PY{c+c1}{\PYZsh{} 误差反向传播, 计算参数更新值(反向传递)}
            \PY{n}{optimizer}\PY{o}{.}\PY{n}{step}\PY{p}{(}\PY{p}{)}        \PY{c+c1}{\PYZsh{} 将参数更新值施加到 net 的 parameters 上(以学习效率0.5优化步骤)}
            
            \PY{c+c1}{\PYZsh{} 每5步plt显示一次,显示神经网络的训练过程}
            \PY{k}{if} \PY{n}{t} \PY{o}{\PYZpc{}} \PY{l+m+mi}{5} \PY{o}{==} \PY{l+m+mi}{0}\PY{p}{:}
                \PY{n}{plt}\PY{o}{.}\PY{n}{cla}\PY{p}{(}\PY{p}{)}
                \PY{n}{plt}\PY{o}{.}\PY{n}{scatter}\PY{p}{(}\PY{n}{x}\PY{o}{.}\PY{n}{data}\PY{o}{.}\PY{n}{numpy}\PY{p}{(}\PY{p}{)}\PY{p}{,} \PY{n}{y}\PY{o}{.}\PY{n}{data}\PY{o}{.}\PY{n}{numpy}\PY{p}{(}\PY{p}{)}\PY{p}{)}\PY{c+c1}{\PYZsh{} 原始数据}
                \PY{n}{plt}\PY{o}{.}\PY{n}{plot}\PY{p}{(}\PY{n}{x}\PY{o}{.}\PY{n}{data}\PY{o}{.}\PY{n}{numpy}\PY{p}{(}\PY{p}{)}\PY{p}{,} \PY{n}{prediction}\PY{o}{.}\PY{n}{data}\PY{o}{.}\PY{n}{numpy}\PY{p}{(}\PY{p}{)}\PY{p}{,} \PY{l+s+s1}{\PYZsq{}}\PY{l+s+s1}{r\PYZhy{}}\PY{l+s+s1}{\PYZsq{}}\PY{p}{,} \PY{n}{lw}\PY{o}{=}\PY{l+m+mi}{5}\PY{p}{)}\PY{c+c1}{\PYZsh{} 预测数据}
                \PY{c+c1}{\PYZsh{} 当前误差}
                \PY{n}{plt}\PY{o}{.}\PY{n}{text}\PY{p}{(}\PY{l+m+mf}{0.5}\PY{p}{,} \PY{l+m+mi}{0}\PY{p}{,} \PY{l+s+s1}{\PYZsq{}}\PY{l+s+s1}{Loss=}\PY{l+s+si}{\PYZpc{}.4f}\PY{l+s+s1}{\PYZsq{}} \PY{o}{\PYZpc{}} \PY{n}{loss}\PY{o}{.}\PY{n}{data}\PY{p}{[}\PY{l+m+mi}{0}\PY{p}{]}\PY{p}{,} \PY{n}{fontdict}\PY{o}{=}\PY{p}{\PYZob{}}\PY{l+s+s1}{\PYZsq{}}\PY{l+s+s1}{size}\PY{l+s+s1}{\PYZsq{}}\PY{p}{:} \PY{l+m+mi}{20}\PY{p}{,} \PY{l+s+s1}{\PYZsq{}}\PY{l+s+s1}{color}\PY{l+s+s1}{\PYZsq{}}\PY{p}{:}  \PY{l+s+s1}{\PYZsq{}}\PY{l+s+s1}{red}\PY{l+s+s1}{\PYZsq{}}\PY{p}{\PYZcb{}}\PY{p}{)}
                \PY{n}{plt}\PY{o}{.}\PY{n}{pause}\PY{p}{(}\PY{l+m+mf}{0.1}\PY{p}{)}  
        
        
        \PY{c+c1}{\PYZsh{} plt.ioff()}
        \PY{c+c1}{\PYZsh{} plt.show()}
            
            
\end{Verbatim}


    \subsection{区分类型(分类)}\label{ux533aux5206ux7c7bux578bux5206ux7c7b}

    \begin{Verbatim}[commandchars=\\\{\}]
{\color{incolor}In [{\color{incolor}18}]:} \PY{o}{\PYZpc{}\PYZpc{}}\PY{k}{classify}.py\PYZpc{}\PYZpc{}
         \PYZsh{}\PYZsh{}\PYZsh{} 建立数据集
         import torch
         from torch.autograd import Variable
         import matplotlib.pyplot as plt
         import torch.nn.functional as F     \PYZsh{} 激励函数都在这
         
         \PYZsh{} 假数据
         n\PYZus{}data = torch.ones(100, 2)         \PYZsh{} 数据的基本形态
         x0 = torch.normal(2*n\PYZus{}data, 1)      \PYZsh{} 类型0 x data (tensor), shape=(100, 2)
         y0 = torch.zeros(100)               \PYZsh{} 类型0 y data (tensor), shape=(100, 1)
         x1 = torch.normal(\PYZhy{}2*n\PYZus{}data, 1)     \PYZsh{} 类型1 x data (tensor), shape=(100, 1)
         y1 = torch.ones(100)                \PYZsh{} 类型1 y data (tensor), shape=(100, 1)
         
         \PYZsh{} 注意 x, y 数据的数据形式是一定要像下面一样 (torch.cat 是在合并数据)
         x = torch.cat((x0, x1), 0).type(torch.FloatTensor)  \PYZsh{} FloatTensor = 32\PYZhy{}bit floating
         \PYZsh{} pytorch规定的标签类型必须是LongTensor
         y = torch.cat((y0, y1), ).type(torch.LongTensor)    \PYZsh{} LongTensor = 64\PYZhy{}bit integer
         
         \PYZsh{} torch 只能在 Variable 上训练, 所以把它们变成 Variable
         x, y = Variable(x), Variable(y)
         
         plt.scatter(x.data.numpy()[:, 0], x.data.numpy()[:, 1], c=y.data.numpy(), s=100, lw=0, cmap=\PYZsq{}RdYlGn\PYZsq{})
         plt.show()
         
         \PYZsh{} 画图
         \PYZsh{} plt.scatter(x.data.numpy(), y.data.numpy())
         \PYZsh{} plt.show()
         
         
         \PYZsh{}\PYZsh{}\PYZsh{} 建立神经网络
         class Net(torch.nn.Module):     \PYZsh{} 继承 torch 的 Module
             def \PYZus{}\PYZus{}init\PYZus{}\PYZus{}(self, n\PYZus{}feature, n\PYZus{}hidden, n\PYZus{}output):
                 super(Net, self).\PYZus{}\PYZus{}init\PYZus{}\PYZus{}()     \PYZsh{} 继承 \PYZus{}\PYZus{}init\PYZus{}\PYZus{} 功能
                 self.hidden = torch.nn.Linear(n\PYZus{}feature, n\PYZus{}hidden)   \PYZsh{} 隐藏层线性输出
                 self.out = torch.nn.Linear(n\PYZus{}hidden, n\PYZus{}output)       \PYZsh{} 输出层线性输出
         
             def forward(self, x):
                 \PYZsh{} 正向传播输入值, 神经网络分析出输出值
                 x = F.relu(self.hidden(x))      \PYZsh{} 激励函数(隐藏层的线性值)
                 x = self.out(x)                 \PYZsh{} 输出值, 但是这个不是预测值, 预测值还需要再另外计算
                 return x
         
         net = Net(n\PYZus{}feature=2, n\PYZus{}hidden=10, n\PYZus{}output=2) \PYZsh{} 几个类别就几个 output
         
         print(net)  \PYZsh{} net 的结构
         
         \PYZsh{}\PYZsh{}\PYZsh{} 训练神经网络
         \PYZsh{} optimizer 是训练的工具
         optimizer = torch.optim.SGD(net.parameters(), lr=0.02)  \PYZsh{} 传入 net 的所有参数, 学习率
         \PYZsh{} 算误差的时候, 注意真实值!不是! one\PYZhy{}hot 形式的, 而是1D Tensor, (batch,)
         \PYZsh{} 但是预测值是2D tensor (batch, n\PYZus{}classes)
         \PYZsh{} 回归问题:loss\PYZus{}func = torch.nn.MSELoss()
         \PYZsh{} 分类问题:特别是多分类:CrossEntropyLoss更加合适,输出的概率,例如[0.1, 0.2, 0.7]
         loss\PYZus{}func = torch.nn.CrossEntropyLoss()
         \PYZsh{} 标签误差形式[0, 0, 1]
         
         plt.ion()   \PYZsh{} 画图
         plt.show()
         
         for t in range(100):
             out = net(x)     \PYZsh{} 喂给 net 训练数据 x, 输出分析值
         
             loss = loss\PYZus{}func(out, y)     \PYZsh{} 计算两者的误差
         
             optimizer.zero\PYZus{}grad()   \PYZsh{} 清空上一步的残余更新参数值
             loss.backward()         \PYZsh{} 误差反向传播, 计算参数更新值
             optimizer.step()        \PYZsh{} 将参数更新值施加到 net 的 parameters 上
             
             \PYZsh{}\PYZsh{}\PYZsh{}\PYZsh{} 画图
             if t \PYZpc{} 2 == 0:
                 plt.cla()
                 \PYZsh{} 过了一道 softmax 的激励函数后的最大概率才是预测值
                 prediction = torch.max(F.softmax(out), 1)[1]
                 pred\PYZus{}y = prediction.data.numpy().squeeze()
                 target\PYZus{}y = y.data.numpy()
                 plt.scatter(x.data.numpy()[:, 0], x.data.numpy()[:, 1], c=pred\PYZus{}y, s=100, lw=0, cmap=\PYZsq{}RdYlGn\PYZsq{})
                 accuracy = sum(pred\PYZus{}y == target\PYZus{}y)/200  \PYZsh{} 预测中有多少和真实值一样
                 plt.text(1.5, \PYZhy{}4, \PYZsq{}Accuracy=\PYZpc{}.2f\PYZsq{} \PYZpc{} accuracy, fontdict=\PYZob{}\PYZsq{}size\PYZsq{}: 20, \PYZsq{}color\PYZsq{}:  \PYZsq{}red\PYZsq{}\PYZcb{})
                 plt.pause(0.1)
         
         plt.ioff()  \PYZsh{} 停止画图
         plt.show()
             
\end{Verbatim}


    \begin{Verbatim}[commandchars=\\\{\}]
UsageError: Cell magic `\%\%classify.py\%\%` not found.

    \end{Verbatim}

    \begin{itemize}
\tightlist
\item
  有两个类型,所以输出层2
\end{itemize}

\texttt{net\ =\ Net(2,\ 10,\ 2)\#\ 输入层:2;隐藏层10;输出层:2}

    \begin{itemize}
\tightlist
\item
  二分类

  \begin{itemize}
  \item
  \item
  \end{itemize}
\item
  三分类

  \begin{itemize}
  \item
  \item
    {[}0, 1, 0{]}:分类为1
  \item
    {[}0, 0, 1{]}:分类为2
  \end{itemize}
\end{itemize}


    % Add a bibliography block to the postdoc
    
    
    
    \end{document}
